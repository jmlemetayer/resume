\documentclass[11pt,a4paper,sans]{moderncv}

% moderncv themes
\moderncvstyle{casual}
\moderncvcolor{grey}
\renewcommand{\familydefault}{\sfdefault}

% character encoding
%\usepackage[utf8]{inputenc}

% adjust the page margins
\usepackage[scale=0.75]{geometry}
\setlength{\hintscolumnwidth}{3cm}

% social
\makeatletter
\newcommand*{\gitlabsocialsymbol}{{\small\faGitlab}~}
\newcommand*{\stackoveflowsocialsymbol}{{\small\faStackOverflow}~}
\RenewDocumentCommand{\social}{O{}O{}m}{
	\ifthenelse{\equal{#2}{}}
	{
		\ifthenelse{\equal{#1}{linkedin}}
			{\collectionadd[linkedin]{socials}
			{\protect\httplink[#3]{linkedin.com/in/#3}}}{}
		\ifthenelse{\equal{#1}{github}}
			{\collectionadd[github]{socials}
			{\protect\httplink[#3]{github.com/#3}}}{}
		\ifthenelse{\equal{#1}{gitlab}}
			{\collectionadd[gitlab]{socials}
			{\protect\httplink[#3]{gitlab.com/#3}}}{}
		\ifthenelse{\equal{#1}{stackoveflow}}
			{\collectionadd[stackoveflow]{socials}
			{\protect\httplink[#3]{stackoverflow.com/users/#3}}}{}
	}
	{\collectionadd[#1]{socials}{\protect\httplink[#3]{#2}}}}
\makeatother

% personal data
\name{Jean-Marie}{LEMETAYER}
\title{Embedded systems engineer}
\address{14 Helsinki street}{35830 Betton}
\mobile{+33 615 327 374}
\email{jeanmarie.lemetayer@gmail.com}
\homepage{jml.bzh}
\social[github]{jmlemetayer}
\social[gitlab]{jmlemetayer}
\social[linkedin]{jmlemetayer}
\social[stackoveflow]{2893600/jml}
\photo[72pt][0.6pt]{id.jpg}

% content
\begin{document}
\makecvtitle

\section{Experience}
\subsection{Vocational}
\cventry{December 2020 \\ Present}
	{Embedded System Engineer}
	{NXP}
	{Rennes}
	{\textit{on behalf of Elsys-Design}}
	{
		\begin{itemize}
		\item Architecture and development of the firmware update
			application in the bootrom of a new, in development,
			\href{https://en.wikipedia.org/wiki/Ultra-wideband}{UWB}
			chip
		\item Development of the unitary test cases for all the protocol
			layers (RTL simulation)
		\item Development of the system test cases for the firmware
			update feature (FPGA target)
		\end{itemize}
		\textit{Git},
		\textit{C}, \textit{Python}
	}
\cventry{April 2019 \\ July 2020}
	{Embedded System Engineer}
	{Savoir-Faire Linux}
	{Rennes}
	{}
	{
		\begin{itemize}
		\item Customer BSP creation and integration
		\item Customer application development
		\item Contributions to the Yocto project
		\end{itemize}
		\textit{Linux},
		\textit{Git},
		\textit{Yocto},
		\textit{C}, \textit{C++}, \textit{Python}
	}
\cventry{August 2016 \\ Mars 2019}
	{Embedded System Engineer}
	{Kerlink}
	{Rennes}
	{\textit{on behalf of Elsys-Design}}
	{
		\begin{itemize}
		\item LoRa related applications maintainer
		\item Semtech LoRa packet forwarders integration
		\item Internal LoRa solutions development
		\end{itemize}
		\textit{Linux},
		\textit{Git},
		\textit{Buildroot}, \textit{Yocto},
		\textit{C}, \textit{C++},
		\textit{Zeromq}, \textit{Protobuf},
		\textit{LoRa}, \textit{IoT}
	}
\cventry{June 2015 \\ June 2016}
	{Embedded System Engineer}
	{Syrlinks}
	{Rennes}
	{\textit{on behalf of Elsys-Design}}
	{
		\begin{itemize}
		\item Industrialization of the
			\href{http://www.refimeve.fr}{REFIMEVE+} project
		\item Development of an STM32F4 based sensor board using FreeRTOS
		\item Iridium driver and ZMODEM protocol implementation
		\item Linux daemon creation on the monitoring PC
		\end{itemize}
		\textit{Stm32},
		\textit{Linux},
		\textit{Git}, \textit{Svn},
		\textit{C}, \textit{Makefile},
		\textit{FreeRTOS}, \textit{libopencm3},
		\textit{Iridium}, \textit{Zmodem}
	}
\cventry{November 2014 \\ June 2015}
	{Embedded System Engineer}
	{Mitsubishi Electric}
	{Rennes}
	{\textit{on behalf of Elsys-Design}}
	{
		\begin{itemize}
		\item Development of an Ethernet scheduler (IEEE 802.1Qbv)
			with a CAN-Ethernet encapsulation (IEEE P1722a)
			on a STM32F4 target using FreeRTOS
		\item In order to fill the Ethernet bandwidth a video stream
			generated by a camera module is used
		\end{itemize}
		\textit{Stm32},
		\textit{Linux},
		\textit{C}, \textit{Makefile},
		\textit{FreeRTOS}, \textit{libopencm3},
		\textit{Ethernet}, \textit{CAN}
	}
\cventry{February 2014 \\ September 2014}
	{Embedded System Integrator}
	{Aviwest}
	{Rennes}
	{\textit{on behalf of Elsys-Design}}
	{
		\begin{itemize}
		\item Embedded System integration
		\item Custom Wi-Fi connection manager development
		\end{itemize}
		\textit{Linux},
		\textit{Svn},
		\textit{Buildroot},
		\textit{C}, \textit{Makefile},
		\textit{Wi-Fi}
	}
\cventry{December 2013 \\ January 2014}
	{Embedded System Engineer}
	{Sagemcom}
	{Paris}
	{\textit{on behalf of Elsys-Design}}
	{
		\begin{itemize}
		\item Wi-Fi integration on a set-top box
		\item Wi-Fi certification adviser
		\end{itemize}
		\textit{Linux},
		\textit{Svn},
		\textit{Buildroot},
		\textit{C},
		\textit{Wi-Fi}
	}
\cventry{April 2012 \\ November 2013}
	{System Integration Engineer - Wi-Fi}
	{Intel Corporation}
	{Nice}
	{\textit{on behalf of Elsys-Design}}
	{
		\begin{itemize}
		\item Android Wi-Fi integration for Intel platforms
		\item Wi-Fi validation and certification
		\item Internal course: IEEE 802.11n
			- Design Details \& Protocol Analysis
		\end{itemize}
		\textit{Linux}, \textit{Android},
		\textit{Git},
		\textit{C}, \textit{C++},
		\textit{Wi-Fi}
	}
\cventry{August 2010 \\ April 2012}
	{Embedded System Engineer}
	{Faiveley Transport}
	{Rennes}
	{\textit{on behalf of Elsys-Design}}
	{
		\begin{itemize}
		\item Development of a train door application layer control
		\item Architecture specification
		\item SIL2 certification
		\end{itemize}
		\textit{Svn},
		\textit{C}
	}

\subsection{Miscellaneous}
\cventry{2010 (6 months)}
	{Android porting on a Tablet PC}
	{Altran}
	{Rennes}
	{\textit{Final year study project}}
	{
		\begin{itemize}
		\item Linux kernel integration on the ST Nomadik 8815 processor
		\item Linux kernel update to support Android
		\item Setting up the Android rootfs
		\end{itemize}
		\textit{Linux}, \textit{Android},
		\textit{Git},
		\textit{C}
	}
\cventry{2009 (2 months)}
	{Xenomai framework study}
	{ENSEIRB-MATMECA}
	{Bordeaux}
	{\textit{Advanced project}}
	{
		\begin{itemize}
		\item Setting up Xenomai on PC
		\item Study of the Xenomai port on embedded platforms
		\end{itemize}
		\textit{Linux}, \textit{C}
	}

\clearpage

\section{Education}
\cventry{2007 - 2010}
	{Master of Engineering}
	{ENSEIRB-MATMECA}
	{Bordeaux}
	{}
	{Electronics - Embedded Systems option}
\cventry{2005 - 2007}
	{CPGE}
	{Chateaubriand High School}
	{Rennes}
	{}
	{
		\begin{itemize}
		\item PCSI (Physics, Chemistry, and Engineering Science)
		\item followed by PSI* (Physics and Engineering Science)
		\end{itemize}
	}

\section{Languages}
\cvitemwithcomment{French}{Native}{}
\cvitemwithcomment{English}{Full professional proficiency}{TOEIC 795 (2009)}

\section{Computer skills}
\cvitem{OS}{GNU/Linux, Debian, Android}
\cvitem{Languages}{C (expert), C++ (advanced)}
\cvitem{Scripts}{Shell (expert), Makefile (expert), Python (advanced)}
\cvitem{Tools}{vim, git, gcc, gdb, valgrind}

\section{Open Source Contributions}
\cvlistdoubleitem{\href{https://git.yoctoproject.org/cgit/cgit.cgi/poky/log/?qt=grep&q=lemetayer}{poky}}
		 {\href{https://git.openembedded.org/openembedded-core/log/?qt=grep&q=lemetayer}{openembedded-core}}
\cvlistdoubleitem{\href{https://git.openembedded.org/bitbake/log/?qt=grep&q=lemetayer}{bitbake}}
		 {\href{https://w1.fi/cgit/hostap/log/?qt=grep&q=lemetayer}{hostap}}
\cvlistdoubleitem{\href{https://github.com/crosstool-ng/crosstool-ng/commits?author=jmlemetayer}{crosstool-ng}}
		 {\href{http://lists.infradead.org/pipermail/linux-arm-kernel/2010-March/012556.html}{linux-arm-kernel}}

\clearpage

\recipient{Company Recruitment team}{Company, Inc.\\123 somestreet\\some city}
\date{January 01, 2000}
\opening{Dear Sir or Madam,}
\closing{Yours faithfully,}
\enclosure[Attached]{curriculum vit\ae{}}
\makelettertitle

Lorem ipsum dolor sit amet, consectetur adipiscing elit.

\makeletterclosing
\end{document}
