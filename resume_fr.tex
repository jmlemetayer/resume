\documentclass[11pt,a4paper,sans]{moderncv}

% moderncv themes
\moderncvstyle{casual}
\moderncvcolor{grey}
\renewcommand{\familydefault}{\sfdefault}

% character encoding
%\usepackage[utf8]{inputenc}

% adjust the page margins
\usepackage[scale=0.75]{geometry}
\setlength{\hintscolumnwidth}{3cm}

% social
\makeatletter
\newcommand*{\gitlabsocialsymbol}{{\small\faGitlab}~}
\newcommand*{\stackoveflowsocialsymbol}{{\small\faStackOverflow}~}
\RenewDocumentCommand{\social}{O{}O{}m}{
	\ifthenelse{\equal{#2}{}}
	{
		\ifthenelse{\equal{#1}{linkedin}}
			{\collectionadd[linkedin]{socials}
			{\protect\httplink[#3]{linkedin.com/in/#3}}}{}
		\ifthenelse{\equal{#1}{github}}
			{\collectionadd[github]{socials}
			{\protect\httplink[#3]{github.com/#3}}}{}
		\ifthenelse{\equal{#1}{gitlab}}
			{\collectionadd[gitlab]{socials}
			{\protect\httplink[#3]{gitlab.com/#3}}}{}
		\ifthenelse{\equal{#1}{stackoveflow}}
			{\collectionadd[stackoveflow]{socials}
			{\protect\httplink[#3]{stackoverflow.com/users/#3}}}{}
	}
	{\collectionadd[#1]{socials}{\protect\httplink[#3]{#2}}}}
\makeatother

% personal data
\name{Jean-Marie}{LEMETAYER}
\title{Ingénieur Systèmes Embarqués}
\address{14 rue d'Helsinki}{35830 Betton}
\mobile{+33 615 327 374}
\email{jeanmarie.lemetayer@gmail.com}
\homepage{jml.bzh}
\social[github]{jmlemetayer}
\social[gitlab]{jmlemetayer}
\social[linkedin]{jmlemetayer}
\social[stackoveflow]{2893600/jml}
\photo[72pt][0.6pt]{id.jpg}

% content
\begin{document}
\makecvtitle

\section{Expérience}
\subsection{Professionnelle}
\cventry{Décembre 2020 \\ Present}
	{Ingénieur Systèmes Embarqués}
	{NXP}
	{Rennes}
	{\textit{au nom de Elsys-Design}}
	{
		\begin{itemize}
		\item Architecture et développement de l'application de mise à
			jour du firmware dans la bootrom d'une nouvelle puce
			\href{https://en.wikipedia.org/wiki/Ultra-wideband}{UWB}
			en cours de développement
		\item Développement des tests unitaires pour les couches
			protocolaires (simulation RTL)
		\item Développement des tests système de la mise à jour du
			firmware (cible FPGA)
		\end{itemize}
		\textit{Git},
		\textit{C}, \textit{Python}
	}
\cventry{Avril 2019 \\ Juillet 2020}
	{Ingénieur Systèmes Embarqués}
	{Savoir-Faire Linux}
	{Rennes}
	{}
	{
		\begin{itemize}
		\item Création et intégration de BSP
		\item Développement d'applications spécifique
		\item Contributions au projet Yocto
		\end{itemize}
		\textit{Linux},
		\textit{Git},
		\textit{Yocto},
		\textit{C}, \textit{C++}, \textit{Python}
	}
\cventry{Août 2016 \\ Mars 2019}
	{Ingénieur Systèmes Embarqués}
	{Kerlink}
	{Rennes}
	{\textit{au nom de Elsys-Design}}
	{
		\begin{itemize}
		\item Mainteneur des applications LoRa
		\item Intégration des packet forwarders LoRa de Semtech
		\item Développement de solutions LoRa internes
		\end{itemize}
		\textit{Linux},
		\textit{Git},
		\textit{Buildroot}, \textit{Yocto},
		\textit{C}, \textit{C++},
		\textit{Zeromq}, \textit{Protobuf},
		\textit{LoRa}, \textit{IoT}
	}
\cventry{Juin 2015 \\ Juin 2016}
	{Ingénieur Systèmes Embarqués}
	{Syrlinks}
	{Rennes}
	{\textit{au nom de Elsys-Design}}
	{
		\begin{itemize}
		\item Industrialisation du projet
			\href{http://www.refimeve.fr}{REFIMEVE+}
		\item Développement d'une carte capteur basée sur un STM32F4
			utilisant FreeRTOS
		\item Implémentation d'un pilote Iridium et du protocole ZMODEM
		\item Création d'un daemon Linux sur un PC de surveillance
		\end{itemize}
		\textit{Stm32},
		\textit{Linux},
		\textit{Git}, \textit{Svn},
		\textit{C}, \textit{Makefile},
		\textit{FreeRTOS}, \textit{libopencm3},
		\textit{Iridium}, \textit{Zmodem}
	}
\cventry{Novembre 2014 \\ Juin 2015}
	{Ingénieur Systèmes Embarqués}
	{Mitsubishi Electric}
	{Rennes}
	{\textit{au nom de Elsys-Design}}
	{
		\begin{itemize}
		\item Développement d'un ordonnanceur Ethernet (IEEE 802.1Qbv)
			avec une encapsulation CAN-Ethernet (IEEE P1722a)
				sur une cible STM32F4 utilisant FreeRTOS
		\item Afin de remplir la bande passante Ethernet, un flux vidéo
			généré par un module caméra est utilisé
		\end{itemize}
		\textit{Stm32},
		\textit{Linux},
		\textit{C}, \textit{Makefile},
		\textit{FreeRTOS}, \textit{libopencm3},
		\textit{Ethernet}, \textit{CAN}
	}
\cventry{Février 2014 \\ Septembre 2014}
	{Embedded System Integrator}
	{Aviwest}
	{Rennes}
	{\textit{au nom de Elsys-Design}}
	{
		\begin{itemize}
		\item Intégration de systèmes embarqués
		\item Développement d'un gestionnaire de connexion Wi-Fi
		\end{itemize}
		\textit{Linux},
		\textit{Svn},
		\textit{Buildroot},
		\textit{C}, \textit{Makefile},
		\textit{Wi-Fi}
	}
\cventry{Décembre 2013 \\ Janvier 2014}
	{Ingénieur Systèmes Embarqués}
	{Sagemcom}
	{Paris}
	{\textit{au nom de Elsys-Design}}
	{
		\begin{itemize}
		\item Intégration du Wi-Fi sur une set-top box
		\item Conseiller pour la certification Wi-Fi
		\end{itemize}
		\textit{Linux},
		\textit{Svn},
		\textit{Buildroot},
		\textit{C},
		\textit{Wi-Fi}
	}
\cventry{Avril 2012 \\ Novembre 2013}
	{Ingénieur Intégration Système - Wi-Fi}
	{Intel Corporation}
	{Nice}
	{\textit{au nom de Elsys-Design}}
	{
		\begin{itemize}
		\item Intégration Wi-Fi Android pour les plateformes Intel
		\item Validation et certification Wi-Fi
		\item Cours interne: IEEE 802.11n
			- Design Details \& Protocol Analysis
		\end{itemize}
		\textit{Linux}, \textit{Android},
		\textit{Git},
		\textit{C}, \textit{C++},
		\textit{Wi-Fi}
	}
\cventry{Août 2010 \\ Avril 2012}
	{Ingénieur Systèmes Embarqués}
	{Faiveley Transport}
	{Rennes}
	{\textit{au nom de Elsys-Design}}
	{
		\begin{itemize}
		\item Développement de la couche applicative du contrôle des
			portes de train
		\item Spécification de l'architecture
		\item Certification SIL2
		\end{itemize}
		\textit{Svn},
		\textit{C}
	}

\subsection{Divers}
\cventry{2010 (6 mois)}
	{Portage d'Android sur un Tablet PC}
	{Altran}
	{Rennes}
	{\textit{Projet de fin d'études}}
	{
		\begin{itemize}
		\item Intégration du kernel Linux pour le processeur ST Nomadik 8815
		\item Modification du kernel pour l'intégration d'Android
		\item Mise en place du rootfs d'Android
		\end{itemize}
		\textit{Linux}, \textit{Android},
		\textit{Git},
		\textit{C}
	}
\cventry{2009 (2 mois)}
	{Étude du noyau Xenomai}
	{ENSEIRB-MATMECA}
	{Bordeaux}
	{\textit{Projet avancé}}
	{
		\begin{itemize}
		\item Mise en \oe{}uvre de Xenomai sur PC
		\item Étude du portage de Xenomai sur une platforme Embarqué
		\end{itemize}
		\textit{Linux}, \textit{C}
	}

\clearpage

\section{Éducation}
\cventry{2007--2010}
	{Diplôme d'Ingénieur}
	{ENSEIRB-MATMECA}
	{Bordeaux}
	{}
	{Électronique - Option Systèmes Embarqués}
\cventry{2005--2007}
	{CPGE}
	{Lycée Chateaubriand}
	{Rennes}
	{}
	{
		\begin{itemize}
		\item PCSI (Physique, Chimie et Science de l'Ingénieur)
		\item suivie de PSI* (Physique, Science de l'Ingénieur)
		\end{itemize}
	}
\cventry{2003--2005}
	{Baccalauréat}
	{Lycée Chateaubriand}
	{Combourg}
	{}
	{Série S (scientifique) - Option Maths}

\section{Langues}
\cvitemwithcomment{Anglais}{Courant}{TOEIC 795 (2009)}

\section{Compétences Informatiques}
\cvitem{OS}{GNU/Linux, Debian, Android}
\cvitem{Langages}{C (expert), C++ (advanced)}
\cvitem{Scripts}{Shell (expert), Makefile (expert), Python (advanced)}
\cvitem{Outils}{vim, git, gcc, gdb, valgrind}

\end{document}
