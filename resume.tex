\documentclass[10pt,a4paper,sans]{moderncv}
\moderncvstyle{classic}
\moderncvcolor{blue}
\renewcommand{\familydefault}{\sfdefault}
\nopagenumbers{}
\usepackage[utf8]{inputenc}
\usepackage[scale=0.75]{geometry}

\firstname{Jean-Marie}
\familyname{Lemetayer}
\title{Ingénieur Systèmes Embarqués}
\address{51 Boulevard Victor Hugo}{06130 Gasse}
\mobile{+33 615 327 374}
\phone{+33 980 690 024}
\email{jeanmarie.lemetayer@gmail.com}
\homepage{github.com/jmlemetayer}
\extrainfo{26 ans}
\photo[64pt][0.4pt]{id}

\begin{document}
\makecvtitle

\section{Expérience}
\subsection{Professionnelle}
\cventry{2012--2013}{Intégration Wi-Fi sur Android}{Intel Corporation}{Sophia Antipolis}{via Elsys-Design}{
	Intégration et développment de la partie conectivité Wi-Fi sur les platformes Intel Android.
	\begin{itemize}
		\item Intégration du Wi-Fi lors des mise à jour d'Android.
		\item Résolution de bugs à différents niveaux.
		\item Certification du Wi-Fi:
		\begin{itemize}
			\item Installation des environements de certification: 11N, P2P, WFD \dots
			\item Développement de l'automatisation (sigma agent).
		\end{itemize}
		\item Création d'un environement de test:
		\begin{itemize}
			\item Architecture du réseau de test.
			\item Installation des serveurs: DHCP, DNS, RADIUS, HTTP, FTP, RADVD \dots
			\item Création d'interfaces web: Gestionnaire d'AP, Configuration IPv6 \dots
		\end{itemize}
	\end{itemize}
}
\cventry{2010--2012}{Développement Système Embarqué}{Faiveley Transport}{Rennes}{via Elsys-Design}{
	Couche applicative de commande d'ouverture de porte de train.
	\begin{itemize}
		\item Implémentation d'une architecture spécifié.
		\item Relecture de code, en vue d'une certification SIL2.
		\item Gestion et développement d'un second projet:
		\begin{itemize}
			\item Spécification de l'architecure.
			\item Dévelopement et documention.
			\item Tests sur maquette.
		\end{itemize}
	\end{itemize}
}

\subsection{Divers}
\cventry{2010 (6 mois)}{Portage d'Android sur un Tablet PC}{Altran}{Rennes}{Projet de fin d'études}{
	\begin{itemize}
		\item Intégration du kernel Linux pour le processeur ST Nomadik 8815.
		\item Modification de drivers Linux. Support de la NAND, du LCD \dots
		\item Patch du kernel pour l'intégration d'Android: Binder, Logger, Low memory killer \dots
		\item Mise en place d'un Rootfs Busybox puis de celui d'android.
	\end{itemize}
}
\cventry{2009 (2 mois)}{Étude du noyau Xenomai}{ENSEIRB-MATMECA}{Bordeaux}{Projet avancé}{
	\begin{itemize}
		\item Mise en \oe{}uvre de Xenomai sur PC x86.
		\item Étude du portage de Xenomai sur un autre processeur.
	\end{itemize}
}
\cventry{2009 (4 mois)}{R\&D dans la domotique}{Delta-Dore SA}{Bonnemain}{Stage}{
	\begin{itemize}
		\item Étude de la technologie RDS avec une carte de développement. Création d'un programme de mise à jour automatique de l'heure via RDS.
		\item Étude du Wi-Fi. Création d'une passerelle Wi-Fi -- PF3D (protocole de domotique).
	\end{itemize}
}

\clearpage

\section{Éducation}
\cventry{2007--2010}{Diplôme d'Ingénieur Électronique}{ENSEIRB-MATMECA}{Bordeaux}{}{Option Systèmes Embarqués.}
\cventry{2005--2007}{CPGE}{Lycée Chateaubriand}{Rennes}{}{PCSI, suivi de PSI.}
\cventry{2003--2005}{Baccalauréat Scientifique}{Lycée Chateaubriand}{Combourg}{}{Option Maths.}

\section{Langues}
\cvitemwithcomment{Anglais}{Courant}{TOEIC 795 (2009)}

\section{Compétences Informatiques}
\cvdoubleitem{Programmation}{C (expert), C++ (avancé)}{Script}{Shell (expert), Python}
\cvdoubleitem{OS}{GNU/Linux, Android}{Web}{HTML/CSS, PHP, Javascipt}
\cvitem{Mes outils}{VI, GIT, SVN, GCC, GDB, Valgrind}

\end{document}
